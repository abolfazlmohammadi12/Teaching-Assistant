%---------------------------------------------------
% کلاس سند: article برای اسناد متنی، با اندازه کاغذ A4 و فونت پایه 16
\documentclass[a4paper,16pt]{article}

% بسته‌های ریاضی برای معادلات و نمادهای پیشرفته
\usepackage{amsmath, amssymb, amstext}

% بسته رنگ برای استفاده از رنگ در متن یا محیط‌ها
\usepackage{xcolor}

% بسته گرافیک برای درج تصاویر
\usepackage{graphicx}

% تنظیمات حاشیه صفحه
\usepackage{geometry}

% بسته لینک‌دهی (هایپرلینک‌ها)
\usepackage{hyperref}

% بسته صفحه‌آرایی برای تنظیم هدر و فوتر
\usepackage{fancyhdr}


% بسته xepersian برای پشتیبانی از زبان فارسی با XeLaTeX
\usepackage{xepersian}

% تعیین فونت فارسی (فایل فونت باید موجود باشد یا در سیستم نصب باشد)
\settextfont{BNAZANIN.TTF}

% تنظیم اندازه حاشیه‌ها از هر طرف برابر با 1 اینچ
\geometry{margin=1in}

% تنظیم سبک fancy برای سربرگ و پاورقی
\pagestyle{fancy}

% سمت چپ هدر: مشخص‌کننده سری تمرین و نام درس
\lhead{تمرین سری ... درس ... }

% وسط هدر: نام دانشکده
\chead{دانشکده فیزیک دانشگاه صنعتی خواجه نصیرالدین طوسی}

% سمت راست هدر: تاریخ روز به‌صورت خودکار
\rhead{\today}

% پایین صفحه: شماره صفحه
\cfoot{\thepage}

% شروع سند
\begin{document}
	
	% صفحه عنوان (بدون هدر و فوتر)
	\thispagestyle{empty}
	\begin{center}
		% لوگوی دانشگاه (باید فایل تصویر در مسیر پروژه باشد)
		\includegraphics[width=0.5\textwidth]{K._N._Toosi_University_of_Technology.png} \\[10pt] 
		
		% عنوان سند
		\textbf{\LARGE عنوان:}\\[20pt]
		
		% نیم‌سال تحصیلی
		\textbf{\LARGE نیم‌سال تحصیلی: }\\[10pt]
		
		% نام مدرس درس
		\textbf{\Large مدرس: }\\[10pt]
		
		% مبحث تمرین 
		\textbf{\Large مبحث تمرین:  }\\[10pt]
		
		% مهلت تحویل تمرین
		\textbf{\Large مهلت تحویل: }
	\end{center}
	
	% صفحه جدید برای ادامه سند
	\newpage
	
	% ایجاد فهرست مطالب به‌صورت خودکار (براساس \section و \subsection و ...)
	\tableofcontents
	\newpage
	
	% شروع بخش سوالات - سوال اول
	\section{سوال اول}
	صورت سوال اول این‌جا نوشته می‌شود.
	\\
	
	% اضافه کردن تصویر در صورت نیاز (در حال حاضر کامنت شده)
	%\begin{figure}[h]
	%    \centering
	%    \includegraphics[width=0.5\textwidth]{FoxitPDFReader_nTeEJVaLnL.png}
	%    \caption{}
	%    \label{fig:square_particles}
	%\end{figure}
	
	% بخش دوم سوالات - سوال امتیازی
	\section{سوال امتیازی}
	صورت سوال امتیازی این‌جا نوشته می‌شود.
	
	
	\vspace{10pt}
	\textbf{موفق باشید.}
	
	% پایان سند
\end{document}
