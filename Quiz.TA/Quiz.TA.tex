% انتخاب کلاس سند article با اندازه فونت ۱۲ و کاغذ A4
\documentclass[12pt,a4paper]{article}

% بسته fontspec برای انتخاب فونت‌های سیستمی (ضروری برای xepersian)
\usepackage{fontspec}

% بسته xepersian برای پشتیبانی از زبان فارسی
\usepackage{xepersian}

% انتخاب فونت فارسی (باید فونت BNAZANIN.TTF در سیستم نصب یا کنار فایل باشد)
\settextfont{BNAZANIN.TTF}

% بسته‌های ریاضی برای نمادها و معادلات
\usepackage{amsmath, amssymb, amstext}

% برای درج تصاویر
\usepackage{graphicx}

% تنظیمات حاشیه صفحه
\usepackage{geometry}

% برای استفاده از رنگ‌ها (مثلاً در متن یا جدول‌ها)
\usepackage{xcolor}

% بسته‌ای برای تنظیم هدر و فوترهای زیبا
\usepackage{fancyhdr}

% تعیین اندازه حاشیه‌ها 
\geometry{margin=1in}

% فعال‌سازی سبک fancy برای صفحه‌آرایی هدر و فوتر
\pagestyle{fancy}

% محتوای هدر سمت چپ: نام آزمون یا تمرین
\lhead{کوییز درس .... }

% محتوای هدر وسط: نام دانشکده
\chead{دانشکده فیزیک دانشگاه صنعتی خواجه نصیرالدین طوسی}

% محتوای هدر سمت راست: تاریخ امروز (خودکار)
\rhead{\today}

% شروع محتوای سند
\begin{document}
	
	% جدول اطلاعات فردی (نام و نام خانوادگی، شماره دانشجویی)
	\begin{tabular}{|p{7cm}|p{7cm}|}
		\hline
		\textbf{نام و نام خانوادگی:} & \textbf{شماره دانشجویی:} \\
		\hline
	\end{tabular}
	
	% بخش سوالات بدون شماره‌گذاری در عنوان
	\section*{سوالات:}
	
	\begin{enumerate}
		% سوال شماره ۱
		\item 
		صورت سوال اول
		
		% اضافه کردن تصویر به سوال (در صورت نیاز)، فعلاً به صورت کامنت
		%\begin{figure}[h]
		%    \centering
		%    \includegraphics[width=0.5\textwidth]{FoxitPDFReader_nTeEJVaLnL.png}
		%    \caption{}
		%    \label{fig:square_particles}
		%\end{figure}
		
	\end{enumerate}
	
	% پایان سند
\end{document}
